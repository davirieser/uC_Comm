\documentclass[12pt,a4paper,fleqn,leqno,openany]{article}

% Package Imports --------------------------------------------------------------

\usepackage{tikz}
\usepackage{tcolorbox}
\usepackage{graphicx}
\usepackage{listings}
\usepackage{xcolor}
\usepackage{lastpage}
\usepackage{fancyhdr}
\usepackage{hyperref}

% Parameter Setup --------------------------------------------------------------

% Define Constant "true"
\newcommand{\true}{true}

% Define automatically filled in Information
\newcommand{\Schueler}{David Rieser}
\newcommand{\Lehrer}{Prof. Roland Richard Lezou}
\newcommand{\Uebungsname}{Micro-Controller-Communication} % Name von der Uebung
\newcommand{\ToC}{true} % Table Of Contents - Inhaltsverzeichnis
\newcommand{\LoF}{true} % List Of Figures - Abbildverzeichnis
\newcommand{\LoL}{true} % List Of Listings - Codeverzeichnis

% Title, Header and Footer Setup -----------------------------------------------
\pagestyle{fancy}
\fancyhf{}

% Define Title-Page
\title{\Uebungsname}
\author{\Schueler\\Betreuer~:~\Lehrer}
% \date{} % When left out fills in current Date automatically

% Define Header and Footer Text
\rhead{5BHEL}
\lhead{\Uebungsname}
\rfoot{Page~\thepage ~of~\pageref{LastPage}}
\lfoot{HTL-Anichstrasse}

% Add Lines under the Header and over the Footer
\renewcommand{\headrulewidth}{1pt}
\renewcommand{\footrulewidth}{1pt}

% Graphics Setup ---------------------------------------------------------------

\graphicspath{{Pictures/}}

% Example Picture
%
% \begin{figure}[ht]
% 	\centering
% 	\includegraphics[width=0.8\textwidth]{Bild.jpg}
% 	\caption{Bildbeschreibung}
%	\label{fig:Bildamrker}
% \end{figure}
%
% Link to Picture
%
% \ref{Referenz:Bild}

% Listings Setup ---------------------------------------------------------------

\lstset{
	alsoletter=-,
    basicstyle=\footnotesize\ttfamily,
    breaklines=true,
	captionpos=b,
	columns=fullflexible,
    frame=trBL,
	language=C,
	numbers=left,
	numbersep=10pt,
	numberstyle=\tiny,
	showstringspaces=false,
	showtabs=true,
	stepnumber=1,
	tab=,
	tabsize=4,
	title=\lstname,
	% Define Comments
    commentstyle=\color{lightgray},
	morecomment=[l][\color{lightgray}]{\\\\},
	morecomment=[l][\color{lightgray}]{//},
	morecomment=[l][\color{lightgray}]{\%},
	morecomment=[s][\color{lightgray}]{/*}{*/},
	morecomment=[s][\color{lightgray}]{\\*}{*\\},
	% Define Strings
	stringstyle=\color{orange},
	morestring=[s][\color{orange}]{\$}{\$}
}

\lstdefinelanguage{LinkerScript}{
	keywords={MEMORY,LENGTH,ORIGIN,SECTIONS},
	keywordstyle=\color{blue}\bfseries,
	ndkeywords={FLASH,RAM},
	ndkeywordstyle=\color{red}\bfseries,
	identifierstyle=\color{black},
	sensitive=false,
 	morestring=[s]{*(}{*)},
 	morestring=[s]{.}{:},
	stringstyle=\color{orange}
}

\lstdefinelanguage{Assembler}{
	keywords={word,global,globl,thumb,weak,thumb_set,thumb_func,end,syntax,cpu},
	keywordstyle=\color{blue}\bfseries,
	ndkeywords={b,bl,bx,mov,movs,CPSIE,CPSID},
	ndkeywordstyle=\color{red}\bfseries,
	identifierstyle=\color{black},
	sensitive=false,
	stringstyle=\color{orange}
}

\lstdefinelanguage{Clang}{
	keywords={include,define,SET_BIT,CLEAR_BIT,WRITE_REG,READ_REG,INTERRUPTS,NO_INTERRUPTS,CHECK_BIT,READ_BIT,WRITE_BIT,CHECK_BIT_MASK,while,for,if,return,write_to_usart,write_bit_to_usart,write_int_bin_to_usart,write_int_hex_to_usart,else},
	keywordstyle=\color{blue}\bfseries,
	ndkeywords={void,uint32_t,uint8_t,int,volatile,char,CARRIAGE_RT,NEWLINE,ifndef,endif,elif},
	ndkeywordstyle=\color{red}\bfseries,
	identifierstyle=\color{black},
	sensitive=false,
	stringstyle=\color{orange}
}

% Example Listing
%
% \begin{lstlisting}[Options]
%	Code
% \end{lstlisting}
%
% Example File Import
%
% \lstinputlisting[Options]{filename}
%
% Define New Language
%
% \lstdefinelanguage{JavaScript}{
%   keywords={typeof, new, true, false, catch, function, return, null, catch, switch, var, if, in, while, do, else, case, break},
%   keywordstyle=\color{blue}\bfseries,
%   ndkeywords={class, export, boolean, throw, implements, import, this},
%   ndkeywordstyle=\color{darkgray}\bfseries,
%   identifierstyle=\color{black},
%   sensitive=false,
%   comment=[l]{//},
%   morecomment=[s]{/*}{*/},
%   commentstyle=\color{purple}\ttfamily,
%   stringstyle=\color{red}\ttfamily,
%   morestring=[b]',
%   morestring=[b]"
% }

% Hyperlink Setup --------------------------------------------------------------

\urlstyle{same}
\hypersetup{
    colorlinks=true,	% Enable Link-Colors
    linkcolor=blue, 	% Color of Internal-Document-Links
    filecolor=magenta, 	% Color of Local-File-Links
    urlcolor=cyan, 		% Color of URL-Links
}

% Hyperref Setup
%
% \label{Marker} 			% To setup the Marker
%
% \url{url}					% Clickable URL
% \href{url}{Link-Text}		% Clickable Text leading to URL
% \ref{Marker}				% To reference the Marker (Returns Number)
% \pageref{Marker}			% To reference the Marker (Returns Number)
% \nameref{Marker}			% To reference the Marker (Returns Name)

% ------------------------------------------------------------------------------
% Document Start ---------------------------------------------------------------
% ------------------------------------------------------------------------------

\begin{document}

% Disable Pagenumbering for Titlepage and Table of Contents
\pagenumbering{gobble}

% Create Title-Page
\begin{titlepage}
	\maketitle
\end{titlepage}

\pagebreak

% Remove Header and Footer for Table of Contents
\thispagestyle{empty}

\ifx\ToC\true
	\renewcommand\contentsname{Table of Contents}
	\tableofcontents
\fi
\ifx\LoF\true
	\let\oldnumberline\numberline
	\renewcommand{\numberline}[1]{\hspace*{-1.5em}}
	\renewcommand\listfigurename{Table of Figures}
	\listoffigures
	\let\numberline\oldnumberline
\fi
\ifx\LoL\true
	\renewcommand\lstlistlistingname{List of Listings}
	\lstlistoflistings
\fi

\pagebreak

\pagenumbering{arabic}

\section{Project Desription}

The goal of the project is to program a STM32F030F4-Chip to send Measurement-Data
to a MAX485 which is a RS485-Transceiver using the STM32's USART3-Interface.\\ \\
The STM32 is connected to the MAX485 through Pins PC10 and PC11.\\ \\
After the STM32 receives a Byte from the MAX485, it is supposed to send back
ten Measurement-Points from the ADC using Interrupts.\\
The MAX485 defines the Connection-Parameters of the USART :
\begin{itemize}
	\item 38400 Baudrate
	\item 8 Data-Bits
	\item 1 Stop-Bit
	\item Odd-Parity
\end{itemize}

\section{Code}

\section{Interrupt Service Routine Setup}

\subsection{Reset Handler}

The Reset-Handler is defined in the Assembler-File and is a Non-Maskable Interrupt with the lowest Priority of all Interrupts.
Low Priority in this case means it will override Interrupts with higher Priority.\\
During Reset the Stack-Pointer has to be reset and then the main-Routine from main.c has to be called.

\lstinputlisting[caption=Assembler Script,language=Assembler,firstline=73,lastline=84]{../../startup.S}

\subsection{Interrupt Service Routine Definitions}

At the End of the Assembler-File are ".weak"-Definitions which are basically Prototypes for Functions which should be defined by other Parts of the Code.

\lstinputlisting[caption=Assembler Script,language=Assembler,firstline=259,lastline=263]{../../startup.S}

The "hang"-Function puts the STM32F030F4 into an endless loop. This is intentional to indicate that an unknown Error has occurred because "hang" should never be called.

\lstinputlisting[caption=Assembler Script,language=Assembler,firstline=97,lastline=99]{../../startup.S}

\pagebreak


\section{Header File}

The Header-File defines a lot of Macros to create more readable and flexible Code.\\
Macros are Key-Value-Pairs that are substited by the C-Preprocessor :

\begin{lstlisting}[language=Clang,caption=Macro Defintion Example]
#define MACRO-NAME VALUE
#define REG_POINTER 0x40000000
#define ADD(a,b) (a + b)

int x = VALUE;
int reg = REG_POINTER;
int result = ADD(x,REG_POINTER);
\end{lstlisting}
\vspace{-25px}
\begin{center}
	\Large compiles to:
\end{center}
\begin{lstlisting}[language=Clang,caption=Substitued Values]
int x = WERT;
int reg = 0x40000000;
int result = (x + 0x40000000);
\end{lstlisting}

\subsection{Address and Register Substitution}

The Header File defines the Addresses of Component-Registers in the Header-File for more Flexibality in the case of a Hardware-Swap.
See full Code at \nameref{subsec:HeaderRef}

\lstinputlisting[caption=Header~File~Address~Definitions~for~Clock-Registers,language=Clang,firstline=80,lastline=90]{../../main.h}

\subsection{Interrupt-Enable and Interrupt-Disable}

The "CPSIE"-, and "CPSID"-Assembler-Instructions which can Enable and Disable Interrupts (except Non-Maskable-Interrupts) are defined in the Assembler-Script and then imported in the Header-File.

\lstinputlisting[caption=Header~File~Prototypes,language=Clang,firstline=6,lastline=8]{../../main.h}

\lstinputlisting[caption=Assembler~Function~Definition,language=Assembler,firstline=107,lastline=119]{../../startup.S}

\subsection{Register Macros}

The Header File defines multiple Macro-Functions for Register-Access to increase Readability.\\
Here an Examle of Functions to Read/Write to Register using Bitmasks :

\lstinputlisting[caption=Header~File~Prototypes,language=Clang,firstline=47,lastline=54]{../../main.h}


\end{document}
