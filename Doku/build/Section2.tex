\section{Memory Setup}

\subsection{Linker Script}

\label{sec:LinkerScript}

The first Step of the Code is the Linker Script which tells the Linker how to
assemble the Binary-File which will be programmed to the STM32F030F4.\\
The following Listing shows the final Code for the Linker Sript:

\lstinputlisting[caption=Linker Script,language=LinkerScript]{../../startup.ld}

\hspace{-15px}"MEMORY" declares the Memory Regions which should be present after programming.
These Memory Regions have several Attributes like a Origin=Start-Address,Length,Readability(r),Writability(w) and Executability(x).\\ \\
The "\_stack"-Symbol defines the Base-Pointer for the Stack so it can be used in the Assembler Script to reset the Stack-Pointer during Reset. See \nameref{sec:VecTabSetup}\\ \\
"SECTIONS" tells the linker where to place the Sections of Codes.
\begin{itemize}
	\item ".text" is the Code Memory (Instructions) and is placed in the Flash.
	\item ".rodata" are Constants and placed in FLASH
	\item ".data" are Variables which are initially assigned a Value not equal 0.
	\item ".bss" are Variables which are initially assigned 0 .
\end{itemize}
More can be read at \\ \url{https://home.cs.colorado.edu/~main/cs1300/doc/gnu/ld_3.html#SEC17}

\pagebreak

\subsection{Vector Table Setup}
\label{sec:VecTabSetup}

The Vector Table is set up in the \nameref{subsec:AssemblerRef} and starts at Memory-Address 4 (Address 0 is occupied by the 4 Byte long Stack Pointer).\\
The Vector Table consists of Addresses which are 4 Bytes long and are loaded into the Program-Counter in the Event that an Interrupt is triggered.\\
See complete \nameref{fig:VecTab}\\
For the complete Code see Section \nameref{subsec:LinkerRef}.

\lstinputlisting[caption=Assembler Script,language=Assembler,firstline=9,lastline=20]{../../startup.S}

\pagebreak
