\section{Project Description}
\label{sec:ProjDesc}

The goal of the project is to program a STM32F030F4-Chip to send Measurement-Data
to a MAX485 which is a RS485-Transceiver using the STM32's USART3-Interface.\\ \\
The STM32 is connected to the MAX485 through Pins PC10 and PC11.\\ \\
After the STM32 receives a Byte from the MAX485, it is supposed to send back
ten Measurement-Points from the ADC using Interrupts.\\
The MAX485 defines the Connection-Parameters of the USART :
\begin{itemize}
	\item 38400 Baudrate
	\item 8 Data-Bits
	\item 1 Stop-Bit
	\item Odd-Parity
\end{itemize}
The Project should be written in C and then compiled using the arm-none-aebi Compiler for Programming using an CP2102.\\ \\
The Project is available on \href{https://github.com/davirieser/uC_Comm}{GitHub}.

\pagebreak

\section{RS485}
\label{sec:RS485}

\subsection{Overview}
\label{subsec:RS_Over}

RS485 is a OSI-Layer 1 Interface, which uses balanced differential Voltages to transmit data.\\
It specifies bidirectional, half-duplex data transmission with up to 32 Receivers and Transmitters each.\\
RS485 requires Shielded Twisted Pair Cables with 120$\Omega$ line impedance and 120$\Omega$ Terminations at both ends of the line.\\
No Maximum Bus Length is specified but it usually ends up at 1.2km at 200kbps or 50m at 10Mbps.\\
RS485 can either be half-duplex using 2 differential ines or full-duplex using 2 sets of 2 differential lines.\\
These 2 lines are called A and B but often called TX+/RX+/D+ and TX-/RX-/D- by manufacturers.

\begin{figure}[!ht]
	\includegraphics[width=\linewidth]{RS485BlockDia.png}
	\caption{RS485 Block Diagram}
	\begin{center}
		\href{http://www.interfacebus.com/Design_Connector_RS485.html#d}{Source : intefacebus.com}
	\end{center}
	\label{fig:RS485BlockDia}
\end{figure}

\pagebreak

\subsection{Functional Description}
\label{subsec:RS_FuncDes}

The Receivers has internal Pull-Up-, and Pull-Down-Resistor to keep the two differential lines U\textsubscript{+} and U\textsubscript{-} of the Receiver at a difference of about 200mV in the Idle-State.\\
When a Transmitter wants to send Data they drive the difference between the
two lines to more than 2V.\\
In the following Picture you can see the Voltage Levels for RS485.
These are differential so how much difference in Potential is between line A and line B.\\
In the gray Area the Line is idle and no data is send.
When a transmitter starts sending it shifts this difference into one of the green
areas.\\
If the Difference is in the top green area (A is greater than B) a Logic Low is send and if it is in the bottom green area (A is lower than B) a Logic High is send.\\

\begin{figure}[!ht]
	\includegraphics[width=\linewidth]{RS485IntefaceLevels.png}
	\caption{RS485 Interface Voltage Levels}
	\begin{center}
		\href{http://www.interfacebus.com/Design_Connector_RS485.html#d}{Source : intefacebus.com}
	\end{center}
	\label{fig:RS485VolLevels}
\end{figure}

\pagebreak

This Voltage difference is sampled as many times as defined by the Baudrate,
includes a Start-Bit and may include a Stop-Bit.\\
In the following Diagram an Example-Communication can be seen:

\begin{figure}[!ht]
	\includegraphics[width=\linewidth]{RS485WaveExample.png}
	\caption{RS485 Example Communication}
	\begin{center}
		\href{https://en.wikipedia.org/wiki/RS-485#Waveform_example}{Source : Wikipedia}
	\end{center}
	\label{fig:RS485ExampComm}
\end{figure}

\subsection{Differences to RS422}
\label{subsec:Diff422}

RS422 is a pretty similar Interface to RS485 with the major difference being that there is no Enable-Line in the RS422 Interface.
This means RS422 Circuits cannot be used in RS485 Circuit but RS485 Circuits can be used in RS422 Circuits.

\pagebreak

\subsection{Connection from STM32 to MAX485}
\label{subsec:RS_Conn}

The original Connection using the PC10 and PC11 Pins could not be accomplished because the USART3-Interface is not available on the STM32F030F4 because of it's smaller TSSOP20-Package.\\
Because of this the Connection is established using the USART1-Interface with Pins PA9 and PA10.

\begin{figure}[!ht]
	\includegraphics[width=\linewidth]{STM32Cap.jpg}
	\caption{STM32F030F4 Capabilities}
	\begin{center}
		\href{https://www.st.com/resource/en/datasheet/stm32f030rc.pdf}{Source : SMT32-Datasheet}
	\end{center}
	\label{fig:PackageCap}
\end{figure}

\begin{figure}
	\includegraphics{STM32Pinout.jpg}
	\caption{STM32F030F4P6 Pinout}
	\begin{center}
		\href{https://www.st.com/resource/en/datasheet/stm32f030rc.pdf}{Source : SMT32-Datasheet}
	\end{center}
	\label{fig:STM32Pinout}
\end{figure}

\pagebreak

\begin{figure}[!ht]
	\includegraphics[width=\linewidth]{STM32DemoBoard.jpg}
	\caption{STM32F030F4P6 Demo Board}
	\begin{center}
		\href{https://stm32-base.org/boards/STM32F030F4P6-STM32F030-DEMO-BOARD-V1.1}{Source : SMT32-Base}
	\end{center}
	\label{fig:DemoBoard}
\end{figure}

\pagebreak
