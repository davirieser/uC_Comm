\section{Program Code}

\subsection{Clock Initialization}

The Clock is setup in the simplest way possible by funneling the High Speed Internal (HSI) Clock through to the USART,ADC and GPIO Pins.

\begin{figure}[!ht]
	\includegraphics[width=\linewidth]{ClockTree.jpg}
	\caption{Clock Tree}
\end{figure}

\pagebreak

The HSI-Clock is a built-in RC-Oscilllator with a frequency of 8MHz.\\
See Register Definition in Refence \nameref{ClockReg}.

\lstinputlisting[caption=Clock Setup,language=CLang,firstline=118,lastline=145]{../../main.c}

\pagebreak

\subsection{Pin Setup for USART}

The GPIO-Pins PA9 and PA10 have to be setup with Alternate Function 1 so they are designated as the RX and TX Pins of the USART1-Component.\\
See Register Definition and Alternate Function Description in Refence \nameref{GpioReg}.

\lstinputlisting[caption=Clock Setup,language=CLang,firstline=46,lastline=116]{../../main.c}

\pagebreak

\subsection{USART Initialization}

The USART-Setup consists of :
\begin{itemize}
	\item Disabling the USART
	\item Setting Baudrate (See \nameref{fig:USARTBRGen})
	\item Setting Word-Length,Stop-Bits and Parity
	\item Enabling Transmitter and Receiver
	\item Enabling USART
	\item Enabling Interrupts
\end{itemize}

\lstinputlisting[caption=Clock Setup,language=CLang,firstline=147,lastline=245]{../../main.c}

\pagebreak

\subsection{ADC Initialization}

The ADC Setup is the trickiest because pretty much every step of the process requires that all Bits in the ADC\_CR-Register are cleared.
That means that after every Step you have wait until the Bits of the previous Step are cleared.\\
A pin also has to be setup as an Anlog Pin for the ADC to measure.\\ \\
The ADC-Setup consists of :
\begin{itemize}
	\item Disable the ADC
	\item Calibrating the ADC-Refernce-Voltage
	\item Configure ADC-Clock
	\item Enabling Interrupts
	\item Selecting ADC-Channel
	\item Enable ADC
\end{itemize}

\lstinputlisting[caption=Clock Setup,language=CLang,firstline=247,lastline=316]{../../main.c}

\pagebreak

\subsection{NVIC Initialization}

Interrupts have to be enabled in the NVIC-Component.
This is done in the Interrupt-Enable-Register by setting the Bit which is equivalent to the Interrupt Number in the \nameref{fig:VecTab} (Bit 12 for ADC and Bit 27 for USART).

\lstinputlisting[caption=Clock Setup,language=CLang,firstline=358,lastline=369]{../../main.c}

\pagebreak

\subsection{Basic Functionality of USART Read and Write}

To Read or Write using the USART you have to check that the USART is ready.
During Writing this means that you have to check that the Transmit-Data-Register-Empty-Bit(TXE) in the ISR-Register is set, indicating that it can be written to. Writing to the Transmit Data Register then clears the TXE-Flag and starts sending the Data.After a Transmission is complete the TC-Flag should be checked to ensure that the Data was send.\\
During Reading this means that you have to check that the Read-Data-Register-Not-Empty-Bit(RXNE) in the ISR-Register is set, indicating that data was received and is now strored in the Read-Data-Register. Reading from the Read-Data-Register then clears the RXNE-Flag.

\lstinputlisting[caption=Clock Setup,language=CLang,firstline=398,lastline=448]{../../main.c}

\pagebreak

\subsection{USART-Interrupt}

The USART-Interrupt has to check which Interrupt has occured using the ISR-Register.\\
If the Read-Data-Not-Empty-Flag is set the USART tells the ADC that it schould send data by changing a global Variable that indicates how many Packets should be sent and enabling the ADC-EOC-Interrupt.\\
See \nameref{sec:ADCInt}

\lstinputlisting[caption=Clock Setup,language=CLang,firstline=680,lastline=743]{../../main.c}

\pagebreak

\subsection{ADC-Interrupt}
\label{sec:ADCInt}

The ADC-Interrupt handles the ADC-Ready-, and End-Of-Conversion-Interrupt.\\
The ADC-Ready-Interrupt disables the ADC-Ready-Interrupt and then tells the ADC to start converting.\\
If the End-of-Conversion-Interrupt is triggered, it indicates that it was enabled by the USART-Interrupt. This means that the ADC-Data should be sent over the USART.
This happens until the "data\_packets\_rem"-global Variable is equal to zero and then the End-Of-Conversion-Interrupt is disabled again. The "data\_packets\_rem"-Variable is decremented every time Data is sent.

\lstinputlisting[caption=Clock Setup,language=CLang,firstline=745,lastline=796]{../../main.c}

\pagebreak
