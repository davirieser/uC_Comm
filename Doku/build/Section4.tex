
\section{Header File}

The Header-File defines a lot of Macros to create more readable and flexible Code.\\
Macros are Key-Value-Pairs that are substited by the C-Preprocessor :

\begin{lstlisting}[language=Clang,caption=Macro Defintion Example]
#define MACRO-NAME VALUE
#define REG_POINTER 0x40000000
#define ADD(a,b) (a + b)

int x = VALUE;
int reg = REG_POINTER;
int result = ADD(x,REG_POINTER);
\end{lstlisting}
\vspace{-25px}
\begin{center}
	\Large compiles to:
\end{center}
\begin{lstlisting}[language=Clang,caption=Substitued Values]
int x = WERT;
int reg = 0x40000000;
int result = (x + 0x40000000);
\end{lstlisting}

\subsection{Address and Register Substitution}

The Header File defines the Addresses of Component-Registers in the Header-File for more Flexibality in the case of a Hardware-Swap.
See full Code at \nameref{subsec:HeaderRef}

\lstinputlisting[caption=Header~File~Address~Definitions~for~Clock-Registers,language=Clang,firstline=80,lastline=90]{../../main.h}

\subsection{Interrupt-Enable and Interrupt-Disable}

The "CPSIE"-, and "CPSID"-Assembler-Instructions which can Enable and Disable Interrupts (except Non-Maskable-Interrupts) are defined in the Assembler-Script and then imported in the Header-File.

\lstinputlisting[caption=Header~File~Prototypes,language=Clang,firstline=6,lastline=8]{../../main.h}

\lstinputlisting[caption=Assembler~Function~Definition,language=Assembler,firstline=107,lastline=119]{../../startup.S}

\subsection{Register Macros}

The Header File defines multiple Macro-Functions for Register-Access to increase Readability.\\
Here an Examle of Functions to Read/Write to Register using Bitmasks :

\lstinputlisting[caption=Header~File~Prototypes,language=Clang,firstline=47,lastline=54]{../../main.h}


\pagebreak
